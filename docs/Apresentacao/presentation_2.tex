%%%%%%%%%%%%%%%%%%%%%%%%%%%%%%%%%%%%%%%%%
% KOMA-Script Presentation
% LaTeX Template
% Version 1.1 (18/10/15)
%
% This template has been downloaded from:
% http://www.LaTeXTemplates.com
%
% Original Authors:
% Marius Hofert (marius.hofert@math.ethz.ch)
% Markus Kohm (komascript@gmx.info)
% Described in the PracTeX Journal, 2010, No. 2
%
% License:
% CC BY-NC-SA 3.0 (http://creativecommons.org/licenses/by-nc-sa/3.0/)
%
%%%%%%%%%%%%%%%%%%%%%%%%%%%%%%%%%%%%%%%%%

%----------------------------------------------------------------------------------------
%	PACKAGES AND OTHER DOCUMENT CONFIGURATIONS
%----------------------------------------------------------------------------------------

\documentclass[
paper=128mm:96mm, % The same paper size as used in the beamer class
fontsize=11pt, % Font size
pagesize, % Write page size to dvi or pdf
parskip=half-, % Paragraphs separated by half a line
]{scrartcl} % KOMA script (article)

\linespread{1.12} % Increase line spacing for readability

%------------------------------------------------
% Colors
\usepackage{xcolor}	 % Required for custom colors
% Define a few colors for making text stand out within the presentation

\definecolor{myblack}{HTML}{594F4F}
\definecolor{mydarkblue}{HTML}{547980}
\definecolor{myblue}{HTML}{45ADA8}
\definecolor{mygreen}{HTML}{9DE0AD}
\definecolor{mygreentwo}{HTML}{92bf9d}
\definecolor{mylightgreen}{HTML}{E5FCC2}
% Use these colors within the presentation by enclosing text in the commands below
\newcommand*{\mygreen}[1]{\textcolor{mygreen}{#1}}
\newcommand*{\mygreentwo}[1]{\textcolor{mygreentwo}{#1}}
\newcommand*{\myblue}[1]{\textcolor{myblue}{#1}}
\newcommand*{\myblack}[1]{\textcolor{myblack}{#1}}
\newcommand*{\mylightgreen}[1]{\textcolor{mylightgreen}{#1}}
%------------------------------------------------

%------------------------------------------------
% Margins
\usepackage[ % Page margins settings
includeheadfoot,
top=3.0mm,
bottom=3.5mm,
left=5.5mm,
right=5.5mm,
headsep=6.5mm,
footskip=8.5mm
]{geometry}
%------------------------------------------------

%------------------------------------------------
% Fonts
\usepackage[brazilian]{babel}
\usepackage[utf8]{inputenc}
\usepackage[T1]{fontenc}	 % For correct hyphenation and T1 encoding
%\usepackage{lmodern} % Default font: latin modern font
%\usepackage{utopia} % Alternative font: utopia
%\usepackage{charter} % Alternative font: low-resolution roman font
%\renewcommand{\familydefault}{\sfdefault} % Sans serif - this may need to be commented to see the alternative fonts
%------------------------------------------------

%------------------------------------------------
% Various required packages
\usepackage{amsthm} % Required for theorem environments
\usepackage{bm} % Required for bold math symbols (used in the footer of the slides)
\usepackage{graphicx} % Required for including images in figures
\usepackage{tikz} % Required for colored boxes
\usepackage{booktabs} % Required for horizontal rules in tables
\usepackage{multicol} % Required for creating multiple columns in slides
\usepackage{lastpage} % For printing the total number of pages at the bottom of each slide

\usepackage{microtype} % Better typography
\usepackage{tocstyle} % Required for customizing the table of contents
%------------------------------------------------

%------------------------------------------------
% Slide layout configuration
\usepackage{scrpage2} % Required for customization of the header and footer
\pagestyle{scrheadings} % Activates the pagestyle from scrpage2 for custom headers and footers
\clearscrheadfoot % Remove the default header and footer
\setkomafont{pageheadfoot}{\normalfont\color{black}\sffamily} % Font settings for the header and footer

% Sets vertical centering of slide contents with increased space between paragraphs/lists
\makeatletter
\renewcommand*{\@textbottom}{\vskip \z@ \@plus 1fil}
\newcommand*{\@texttop}{\vskip \z@ \@plus .5fil}
\addtolength{\parskip}{\z@\@plus .25fil}
\makeatother

% Remove page numbers and the dots leading to them from the outline slide
\makeatletter
\newtocstyle[noonewithdot]{nodotnopagenumber}{\settocfeature{pagenumberbox}{\@gobble}}
\makeatother
\usetocstyle{nodotnopagenumber}

\AtBeginDocument{\renewcaptionname{brazilian}{\contentsname}{\Large Sumário}} % Change the name of the table of contents
%------------------------------------------------

%------------------------------------------------
% Header configuration - if you don't want a header remove this block
\ihead{
\hspace{-2mm}
\begin{tikzpicture}[remember picture,overlay]
\node [xshift=\paperwidth/2,yshift=-\headheight] (mybar) at (current page.north west)[rectangle,fill,inner sep=0pt,minimum width=\paperwidth,minimum height=\headheight,top color=myblue!64,bottom color=myblue]{}; % Colored bar
%\node[below of=mybar,yshift=3.3mm,rectangle,shade,inner sep=0pt,minimum width=128mm,minimum height =1.5mm,top color=black!50,bottom color=white]{}; % Shadow under the colored bar
shadow
\end{tikzpicture}} % Header text defined by the \runninghead command below and colored white for contrast
%------------------------------------------------

%------------------------------------------------
% Footer configuration
\setlength{\footheight}{8mm} % Height of the footer
\addtokomafont{pagefoot}{\footnotesize} % Small font size for the footnote

\ifoot{% Left side
\hspace{-2mm}
\begin{tikzpicture}[remember picture,overlay]
\node [xshift=\paperwidth/2,yshift=\footheight] at (current page.south west)[rectangle,fill,inner sep=0pt,minimum width=\paperwidth,minimum height=3pt,top color=mygreen,bottom color=mygreen]{}; % Green bar
\end{tikzpicture}
\myauthor\ \raisebox{0.2mm}{$\bm{\vert}$}\ \myuni % Left side text
}

\ofoot[\pagemark/\pageref{LastPage}\hspace{-2mm}]{\pagemark/\pageref{LastPage}\hspace{-2mm}} % Right side
%------------------------------------------------
\usepackage{csquotes}
\usepackage[backend=bibtex]{biblatex}


\bibliography{sample}
%------------------------------------------------
% Section spacing - deeper section titles are given less space due to lesser importance
\usepackage{titlesec} % Required for customizing section spacing
\titlespacing{\section}{0mm}{0mm}{0mm} % Lengths are: left, before, after
\titlespacing{\subsection}{0mm}{0mm}{-1mm} % Lengths are: left, before, after
\titlespacing{\subsubsection}{0mm}{0mm}{-2mm} % Lengths are: left, before, after
\setcounter{secnumdepth}{0} % How deep sections are numbered, set to no numbering by default - change to 1 for numbering sections, 2 for numbering sections and subsections, etc
%------------------------------------------------

%------------------------------------------------
% Theorem style
\newtheoremstyle{mythmstyle} % Defines a new theorem style used in this template
{0.5em} % Space above
{0.5em} % Space below
{} % Body font
{} % Indent amount
{\sffamily\bfseries} % Head font
{} % Punctuation after head
{\newline} % Space after head
{\thmname{#1}\ \thmnote{(#3)}} % Head spec
	
\theoremstyle{mythmstyle} % Change the default style of the theorem to the one defined above
\newtheorem{theorem}{Theorem}[section] % Label for theorems
\newtheorem{remark}[theorem]{Remark} % Label for remarks
\newtheorem{algorithm}[theorem]{Algoritmo} % Label for algorithms
\makeatletter % Correct qed adjustment
%------------------------------------------------

%------------------------------------------------
% The code for the box which can be used to highlight an element of a slide (such as a theorem)
\newcommand*{\mybox}[2]{ % The box takes two arguments: width and content
\par\noindent
\begin{tikzpicture}[mynodestyle/.style={rectangle,draw=mygreen,thick,inner sep=2mm,text justified,top color=white,bottom color=white,above}]\node[mynodestyle,at={(0.5*#1+2mm+0.4pt,0)}]{ % Box formatting
\begin{minipage}[t]{#1}
#2
\end{minipage}
};
\end{tikzpicture}
\par\vspace{-1.3em}}
%------------------------------------------------

%----------------------------------------------------------------------------------------
%	PRESENTATION INFORMATION
%----------------------------------------------------------------------------------------

\newcommand*{\mytitle}{Implementação de algoritmo de\\ consultas em janelas} % Title
\newcommand*{\runninghead}{Running Head} % Running head displayed on almost all slides
\newcommand*{\myauthor}{Mateus Barros Rodrigues} % Presenters name(s)
\newcommand*{\mydate}{\today} % Presentation date
\newcommand*{\myuni}{IME - USP} % University or department

%----------------------------------------------------------------------------------------

\begin{document}

\graphicspath{{./figuras/}}

%----------------------------------------------------------------------------------------
%	TITLE SLIDE
%----------------------------------------------------------------------------------------

% Title slide - you may have to tweak a few of the numbers if you wish to make changes to the layout
\thispagestyle{empty} % No slide header and footer
\begin{tikzpicture}[remember picture,overlay] % Background box
\node [xshift=\paperwidth/2,yshift=\paperheight/2] at (current page.south west)[rectangle,fill,inner sep=0pt,minimum width=\paperwidth,minimum height=\paperheight/2,top color=mydarkblue,bottom color=mydarkblue]{}; % Change the height of the box, its colors and position on the page here
\end{tikzpicture}
% Text within the box
\begin{flushright}
\vspace{0.3cm}
\color{white}\sffamily
{\bfseries\Large\mytitle\par} % Title
\vspace{0.2cm}
\normalsize
\myauthor\\ % Author name
\tiny
mateus.barros.rodrigues@usp.br \par
\normalsize
Orientador: Prof. Dr. Carlos Eduardo Ferreira\\ % Date
\tiny
cef@ime.usp.br
\vfill
\end{flushright}

\clearpage


%----------------------------------------------------------------------------------------
%	TABLE OF CONTENTS
%----------------------------------------------------------------------------------------

%% \thispagestyle{empty} % No slide header and footer

%% \small\tableofcontents % Change the font size and print the table of contents - it may be useful to shrink the font size further if the presentation is full of sections
%% % To exclude sections/subsections from the table of contents, put an asterisk after \(sub)section like so: \section*{Section Name}

%% \clearpage

%----------------------------------------------------------------------------------------
%	PRESENTATION SLIDES
%----------------------------------------------------------------------------------------

%\section{Displaying Text}

\thispagestyle{empty} % No slide header and footer

\begin{tikzpicture}[remember picture,overlay] % Background box
\node [xshift=\paperwidth/2,yshift=\paperheight/2] at (current page.south west)[rectangle,fill,inner sep=0pt,minimum width=\paperwidth,minimum height=\paperheight/2,top color=mydarkblue,bottom color=mydarkblue]{}; % Change the height of the box, its colors and position on the page here
\end{tikzpicture}
% Text within the box
\begin{flushleft}
\vspace{0.6cm}
\color{white}\sffamily
{\bfseries\Huge O problema a ser resolvido\par} % Request for questions text
\vfill
\end{flushleft}


\clearpage

%------------------------------------------------

%\subsection{O problema a ser resolvido}

\centering
\includegraphics[scale=0.35]{1}
\begin{flushleft}
  \par Dado um conjunto de segmentos não-intersectantes \\no espaço.
\end{flushleft}
\clearpage

\includegraphics[scale=0.35]{2}
\begin{flushleft}
  \par Quais segmentos intersectam com ou estão numa dada\\ janela de lados paralelos?
\end{flushleft}
\clearpage

%------------------------------------------------

%\subsection{Bullet Points and Numbered Lists}

\begin{flushleft}
  \begin{itemize}
  \item O jeito ingênuo de resolver isso é muito lento.
  \item Portanto, vamos separar esse conjunto em estruturas que nos permitam realizar consultas de forma eficiente.
  \end{itemize}
\end{flushleft}
\clearpage

%------------------------------------------------

\thispagestyle{empty} % No slide header and footer

\begin{tikzpicture}[remember picture,overlay] % Background box
\node [xshift=\paperwidth/2,yshift=\paperheight/2] at (current page.south west)[rectangle,fill,inner sep=0pt,minimum width=\paperwidth,minimum height=\paperheight/2,top color=mydarkblue,bottom color=mydarkblue]{}; % Change the height of the box, its colors and position on the page here
\end{tikzpicture}
% Text within the box
\begin{flushleft}
\vspace{0.6cm}
\color{white}\sffamily
{\bfseries\Huge Estruturas e algoritmos\par} % Request for questions text
\vfill
\end{flushleft}

\clearpage
%------------------------------------------------

\begin{flushleft}
  Organizaremos os segmentos e seus respectivos pontos extremos em 4 estruturas:
\end{flushleft}
  
\clearpage

%------------------------------------------------

\begin{flushleft}
  Organizaremos os segmentos e seus respectivos pontos extremos em 4 estruturas:

  \begin{itemize}
  \item 2 \textbf{\myblack{árvores limite com camadas}}
  \end{itemize}
\end{flushleft}
  
\clearpage

%------------------------------------------------

\begin{flushleft}
  Organizaremos os segmentos e seus respectivos pontos extremos em 4 estruturas:

  \begin{itemize}
  \item 2 \textbf{\myblack{árvores limite com camadas}}
    \begin{itemize}
      \item Tempo de construção: $\mathcal{O}(n \log n)$.
      \item Consumo de espaço: $\mathcal{O}(n \log n)$.
    \end{itemize}
  \end{itemize}
\end{flushleft}

\clearpage


%------------------------------------------------


\begin{flushleft}
  Organizaremos os segmentos e seus respectivos pontos extremos em 4 estruturas:

  \begin{itemize}
  \item 2 árvores limite com camadas
  \item 2 \textbf{\myblack{árvores de segmentos}}
  \end{itemize}
\end{flushleft}
  
\clearpage
%------------------------------------------------

\begin{flushleft}
  Organizaremos os segmentos e seus respectivos pontos extremos em 4 estruturas:

  \begin{itemize}
  \item 2 árvores limite com camadas
  \item 2 \textbf{\myblack{árvores de segmentos}} (uma horizontal e uma vertical)
  \end{itemize}
\end{flushleft}
  
\clearpage

%------------------------------------------------

\begin{flushleft}
  Organizaremos os segmentos e seus respectivos pontos extremos em 4 estruturas:

  \begin{itemize}
  \item 2 árvores limite com camadas
  \item 2 \textbf{\myblack{árvores de segmentos}} (uma horizontal e uma vertical)
    \begin{itemize}
      \item Tempo de construção: $\mathcal{O}(n \log^2 n)$.
      \item Consumo de espaço: $\mathcal{O}(n \log n)$.
    \end{itemize}
  \end{itemize}
\end{flushleft}
  
\clearpage
%------------------------------------------------

%------------------------------------------------

\includegraphics[width=\textwidth,height=0.6\textheight]{3}
\begin{flushleft}
\tiny Fonte: Imagem de Álvaro J. P. Franco, \emph{Dissertação de mestrado} \cite{junio09:MSc}
\normalsize
\end{flushleft}
\clearpage
\clearpage
\includegraphics[width=\textwidth,height=0.86\textheight]{4}
\begin{flushleft}
\tiny Fonte: Imagem de Alper Üngör, \emph{Lecture 15: Windowing queries} \cite{site3}
\normalsize
\end{flushleft}
\clearpage
\clearpage
%------------------------------------------------
%------------------------------------------------

\thispagestyle{empty} % No slide header and footer

\begin{tikzpicture}[remember picture,overlay] % Background box
\node [xshift=\paperwidth/2,yshift=\paperheight/2] at (current page.south west)[rectangle,fill,inner sep=0pt,minimum width=\paperwidth,minimum height=\paperheight/2,top color=mydarkblue,bottom color=mydarkblue]{}; % Change the height of the box, its colors and position on the page here
\end{tikzpicture}
% Text within the box
\begin{flushleft}
\vspace{0.6cm}
\color{white}\sffamily
{\bfseries\Huge Realizando a consulta\par} % Request for questions text
\vfill
\end{flushleft}

\clearpage
%------------------------------------------------

\begin{flushleft}
  A consulta será dividida em 5 etapas:\\    
\end{flushleft}
\clearpage
%------------------------------------------------
\begin{flushleft}
  A consulta será dividida em 5 etapas:\\
  \small
  \begin{enumerate}
  \item Encontramos todos os segmentos com o ponto esquerdo dentro da janela. 
  \end{enumerate}
\end{flushleft}
\clearpage
%------------------------------------------------
\begin{flushleft}
  A consulta será dividida em 5 etapas:\\
  \small
  \begin{enumerate}
  \item Encontramos todos os segmentos com o ponto esquerdo dentro da janela.
  \item Encontramos todos os segmentos com o ponto direito dentro da janela. 
  \end{enumerate}
\end{flushleft}
\clearpage
%------------------------------------------------
\begin{flushleft}
  A consulta será dividida em 5 etapas:\\
  \small
  \begin{enumerate}
  \item Encontramos todos os segmentos com o ponto esquerdo dentro da janela.
  \item Encontramos todos os segmentos com o ponto direito dentro da janela.
  \item Encontramos todos os segmentos que intersectam com o lado esquerdo da janela.
  \end{enumerate}
\end{flushleft}
\clearpage
%------------------------------------------------
\begin{flushleft}
  A consulta será dividida em 5 etapas:\\
  \small
  \begin{enumerate}
  \item Encontramos todos os segmentos com o ponto esquerdo dentro da janela.
  \item Encontramos todos os segmentos com o ponto direito dentro da janela.
  \item Encontramos todos os segmentos que intersectam com o lado esquerdo da janela.
  \item Encontramos todos os segmentos que intersectam com o lado direito da janela.
  \end{enumerate}
\end{flushleft}
\clearpage
%------------------------------------------------
\begin{flushleft}
  A consulta será dividida em 5 etapas:\\
  \small  
  \begin{enumerate}
  \item Encontramos todos os segmentos com o ponto esquerdo dentro da janela.
  \item Encontramos todos os segmentos com o ponto direito dentro da janela.
  \item Encontramos todos os segmentos que intersectam com o lado esquerdo da janela.
  \item Encontramos todos os segmentos que intersectam com o lado direito da janela.
  \item Encontramos todos os segmentos que intersectam com o lado superior da janela.
  \end{enumerate}
\end{flushleft}
\clearpage
\clearpage
%------------------------------------------------
  \begin{itemize}
  \item Complexidade das consultas 1 e 2: $\mathcal{O}(\log n + k)$.\\(com \emph{cascateamento fracionário})
  \item Complexidade das consultas 3,4 e 5: $\mathcal{O}(\log^2 n + k)$.\\
  \end{itemize}
  Onde $k$ é o número de elementos na resposta.

\clearpage
%------------------------------------------------

\includegraphics[width=\textwidth,height=0.6\textheight]{3}
\begin{flushleft}
\tiny Fonte: Imagem de Álvaro J. P. Franco, \emph{Dissertação de mestrado} \cite{junio09:MSc}
\normalsize
\end{flushleft}
\clearpage
%------------------------------------------------
\thispagestyle{empty} % No slide header and footer
\printbibliography

\clearpage
\clearpage
%------------------------------------------------

\thispagestyle{empty} % No slide header and footer

\begin{tikzpicture}[remember picture,overlay] % Background box
\node [xshift=\paperwidth/2,yshift=\paperheight/2] at (current page.south west)[rectangle,fill,inner sep=0pt,minimum width=\paperwidth,minimum height=\paperheight/2,top color=mydarkblue,bottom color=mydarkblue]{}; % Change the height of the box, its colors and position on the page here
\end{tikzpicture}
% Text within the box
\begin{flushright}
\vspace{0.6cm}
\color{white}\sffamily
{\bfseries\Huge Perguntas?\par} % Request for questions text
\vfill
\end{flushright}

\clearpage
\clearpage

\thispagestyle{empty} % No slide header and footer
\begin{tikzpicture}[remember picture,overlay] % Background box
\node [xshift=\paperwidth/2,yshift=\paperheight/2] at (current page.south west)[rectangle,fill,inner sep=0pt,minimum width=\paperwidth,minimum height=\paperheight/2,top color=mydarkblue,bottom color=mydarkblue]{}; % Change the height of the box, its colors and position on the page here
\end{tikzpicture}
% Text within the box
\begin{flushright}
\vspace{0.3cm}
\color{white}\sffamily
{\bfseries\Large\mytitle\par} % Title
\vspace{0.2cm}
\normalsize
\myauthor\\ % Author name
\tiny
mateus.barros.rodrigues@usp.br \par
\normalsize
Orientador: Prof. Dr. Carlos Eduardo Ferreira\\ % Date
\tiny
cef@ime.usp.br
\vfill
\end{flushright}


%----------------------------------------------------------------------------------------

\end{document}

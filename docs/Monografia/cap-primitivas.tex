%% ------------------------------------------------------------------------- %%
\chapter{Defini\c{c}\~oes e Primitivas}
\label{cap:definicoes}

\section{Primitivas geom�tricas}
Explicaremos a seguir algumas das no��es fundamentais que ser�o utilizadas ao longo do trabalho,

\subsection{Pontos e Segmentos}
Neste trabalho trataremos basicamente com pontos e segmentos de reta no espa�o~($\mathbb{R}~\text{e}~\mathbb{R}^2$). Sejam
$x,y \in \mathbb{R} $ definimos um ponto no $\mathbb{R}^2$ como um par $ p = (x,y) $. Um segmento $s$ � da forma
$ s = \overline{(x_1,y_1)(x_2,y_2)}$ onde $u = (x_1,y_1)$ e $v = (x_2,y_2)$ s�o pontos chamados de pontos extremos de $s$. 
\subsection{Compara��es}
\par
Uma outra no��o que ser� usada copiosamente ao longo desta monografia � a no��o de desigualdade em rela��o a uma dada coordenada. Sejam $u,v$ pontos, dizemos que $u \leq_x v $ caso $x(u) < x(v)$ ou $ x(u) = x(v)$ e $y(u) \leq y(v)$ ( Simetricamente definido para desigualdades em rela��o � coordenada $y$), ou seja, sempre comparamos primeiro a coordenada de maior interesse e desempatamos pela segunda coordenada nas compara��es.


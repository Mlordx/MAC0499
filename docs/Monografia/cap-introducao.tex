%% ------------------------------------------------------------------------- %%
\chapter{Introdu��o}
\label{cap:introducao}

Neste trabalho de conclus�o de curso foi abordado o problema de \emph{consultas de segmentos em janelas}, um problema de
\emph{buscas em intervalos ortogonais}, que � um dos t�picos fundamentais da �rea de geometria computacional.\par
Dado um conjunto $S$ de segmentos no espa�o ( Seja no $\mathbb{R}\ ,\ \mathbb{R}^{2}$, etc. ) e uma janela W de lados paralelos,
queremos responder rapidamente a seguinte pergunta: \emph{quais segmentos de $S$ est�o contidos na ou intersectam a janela W?}
\par
Este trabalho foi baseado em \emph{Consultas de segmentos em janelas: algoritmos e estruturas de dados} de \citet{junio09:MSc},
portanto seguiremos a mesma divis�o do problema que foi proposta nessa disserta��o: Encontrar pontos contidos em janelas e achar
todos os segmentos que intersectam com um dado segmento ( Horizontal ou vertical ). Seguiremos tamb�m a mesma divis�o de
cap�tulos: Primeiramente apresentaremos defini��es e primitivas geom�tricas, dedicaremos um cap�tulo para falar de consultas de pontos em janelas, um para falar de encontrar intersec��o de
segmentos e finalmente um onde agregaremos esses algoritmos para resolver o problema proposto. Todo o c�digo desenvolvido foi
escrito em linguagem \emph{python} e est� dispon�vel no \href{http://github.com/mlordx/MAC0499/}{gitHub}.



%%%%%%%%%%%%%%%%%%%%%%%%%%%%%%%%%%%%%%%%%%%%%%%%%%
%-
%-
%-
%%%%%%%%%%%%%%%%%%%%%%%%%%%%%%%%%%%%%%%%%%%%%%%%%%
